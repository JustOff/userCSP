\section{Introduction}
\label{sec:background}

The web browsers security model is rooted in the same-origin policy
(SOP)~\cite{sop}, which isolates resources of one origin from
others. However, attackers still able to subvert the SOP by injecting
malicious contents into the web site such as Cross-site scripting
(XSS) attacks~\cite{xss}. Cross-site Scripting (XSS) attacks in web
applications are considered a major threat. It allows an attacker to
conduct a wide range of potential attacks such as, session hijacking,
stealing of sensitive data and passwords, creation of self-propagating
JavaScript worms, etc. To mitigate XSS attacks, Content security
policy (CSP)~\cite{csp} provide website administrators with a way to
enforce content restrictions at client side.

Content Security Policy is a declarative policy that restricts what
content can be loaded on a web page.  Its primary purpose is to
mitigate Cross-Site Scripting vulnerabilities.  The core issue
exploited by Cross-Site Scripting (XSS) attacks is the lack of
knowledge in web browser’s to distinguish between content that’s
intended to be part of web application, and content that’s been
maliciously injected into web application.  To address this problem,
CSP defines the {\tt Content-Security-Policy} HTTP header that allows
web application developers to create a whitelist of sources of trusted
content, and instruct the client browsers to only execute or render
resources from those sources.  However, it is often difficult for
developers to write a comprehensive Content Security Policy for their
website.  They may worry about their page breaking because
unanticipated but necessary content is blocked.  They may not be able
to easily change the headers their site is sending when these
situations occur, which makes it difficult for them to try different
policies until they find one that is the most restrictive for their
page without breaking site functionality.

\codename changes this!  A developer or user can now view the current
policy set by their site and add their own policy.  They can then
choose whether to apply their policy or the sites policy.  Moreover,
they choose to combine these policies and apply the combined policy.
When combining policies, they have an option to choose from the
strictest subset of the two, or the most open subset.  They can test
their site out with their policy set and tweak it until they have one
that works. 

\codename also allows automatic inference of content security policy
for a website. Automatically inferred CSP policy for a website helps
web developers to figure out what CSP rules to set for their site by
giving them the strictest possible policy they could apply without
breaking the current page. To infer the CSP policy for a website,
\codename analyzes the content on the current web page and recommends
a content security policy (CSP) based on the types of content and
sources of that content.  \codename provides this inferred policy to
developers in the proper syntax for the CSP header, so all a developer
needs to do is tack on the header name and add it to their
site. Moreover, it allows users to enforce inferred policy on the
website. Furthermore, \codename allows savvy users to be able to
voluntarily specify their own Content Security Policy (CSP) for
websites that may not have implemented CSP.

In summary, this poster makes following contributions:

\begin {itemize}

\item We present an automated approach, \codename, for writing
  content security policy (CSP) for a website.

\item \codename allows savvy users to be able to voluntarily specify
  their own Content Security Policy (CSP) for websites that may not
  have implemented CSP.

\item We implemented a prototype of our approach in a Firefox extension.

\end{itemize}
